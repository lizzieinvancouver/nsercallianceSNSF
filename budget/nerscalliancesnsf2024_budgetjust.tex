\documentclass[12pt,oneside]{article}
\renewcommand{\baselinestretch}{1.8}
% \renewcommand*{\thefootnote}{\fnsymbol{footnote}}
\usepackage{sectsty,setspace} 
\usepackage[top=2.1cm, bottom=2.1cm, left=2.1cm, right=2.1cm]{geometry} 
\usepackage{graphicx}
\usepackage{epstopdf}
\usepackage{amsmath,latexsym,amssymb,wasysym}
\usepackage{cite}
\usepackage{citesupernumber}
%\usepackage{citecollapse}
\usepackage{hyperref}
\usepackage{float}
\floatstyle{plaintop}
\restylefloat{table}
\usepackage[table]{xcolor} % http://ctan.org/pkg/xcolor
\usepackage{mathptmx} % closest to Times New Roman (https://tex.stackexchange.com/questions/153168/how-to-set-document-font-to-times-new-roman-by-command)

\setlength\parindent{0pt} % no indents throughout

\setlength{\abovecaptionskip}{1pt}
\setlength{\belowcaptionskip}{-10pt}
\usepackage[font={small}]{caption}

\parskip=5pt
\pagenumbering{arabic}
\pagestyle{plain}
% squeeze spacey
\linespread{0.99}
\addtolength{\itemsep}{-0.05in}
\usepackage{multicol}
 
\newenvironment{smitemize}{
\begin{itemize}
  \setlength{\itemsep}{1pt}
  \setlength{\parskip}{0pt}
  \setlength{\parsep}{0pt}}
{\end{itemize}
}

\usepackage{fancyhdr}
\pagestyle{fancy}
\fancyhead[LO]{NSERC Alliance Intl 2024}
\fancyhead[RO]{E. M. Wolkovich}

\newcommand{\Section}[1]{\vspace{-8pt}\section{\hskip -1em.~~#1}\vspace{-3pt}} 


%%%%%%%%%%%%%%%%%%%
%% To do notes %%
%%%%%%%%%%%%%%%%%%%

\begin{document}
% \bibliographystyle{/Users/Lizzie/Documents/EndnoteRelated/Bibtex/styles/ajhg}
\renewcommand{\labelitemi}{$\textendash$}
\begin{center}
{\bf Budget justification: \\How temporal and spatial mosaics of reproduction determine forest communities}
\vspace{-0.5ex}
\end{center}
{\bf Contributions from partner organization}\\
SNSF has funded my collaborator, Janneke Hille Ris Lambers, at 450,000 CHF (719,740 CAD given a 1:1.60 exchange rate), which will support salaries for a postdoc to oversee soil pathogen work, undergraduate researchers (Hilfassistenten), travel for field work---including collecting data in Switzerland for WP1-WP2---and supplies for soil pathogen studies, as well as sequencing of 500 Swiss soil samples. Dr. Hille Ris Lambers will use funds from other existing grants to support field assistant time (as needed) and one ETH-based PhD student (both of these were originally proposed to be partially covered by the SNSF grant, but will now be covered by other existing sources of funding given that the SNSF award amount has been adjusted)

{\bf Salaries}\\
I request support for 1 MSc and 8-22 undergraduate students (an UBC-based PhD student on this project is supported through other existing sources of funding that I hold). I expect a minimum of 8  undergraduates will be supported, up to a maximum of 22 if we have different students each summer and academic year. My lab, however, has a track record of most students staying in the lab for multiple years, which allows students to gain more in-depth skills and follow the science from beginning to later stages. Thus I expect this project will support  8-12 different individual students over three years. 

The requested amount of \$34,000 for a MSc student represents a base salary of \$27,000 -- consistent with the current tri-agencies master’s scholarships stipend, to which our Faculty asks us to match for all MSc students plus \$7,000 to cover UBC tuition and fees. In our Faculty tuition and fees are paid by the students, while in other Faculties at UBC the departments or faculty covers these costs; thus for equity in salary across the university and to offset the high cost of living in Vancouver, my lab also covers these costs. 

Undergraduate student support is requested in all years of the project to assist in field work at Mount Rainier (also called Mount Tahoma), soil pathogen studies, as well as data cleaning, management and analysis. The current minimum for a Project Worker (the lowest level Work-Learn position to hire students into at UBC) is roughly \$19 per hour, thus I assume \$22 per hour (as the base pay has increased roughly \$2 each of the last two years). I assume this cost will be offset Work-Learn funding (\$9 per hour), and other research awards (e.g. NSERC USRA, which covers currently 60\% of total summer costs for one undergraduate). Thus I estimate \$13 hourly rate for undergraduate students across all years. 

Total costs and hours for undergraduate students: In Year 1, 4 students will each work 4 weeks in the field in the summer  (35 hours per week, per student) and 24 weeks in the lab for the soil pathogen study at roughly 20 hour per week on average (some weeks of prep in the summer will be higher, with lower hours each week during term when the focus will be on monitoring), for a total of 2480 undergraduate hours (4 people x 4 weeks in the field x 35 hours per week plus 4 people x 24 weeks in the lab x 20 hours on average per week) in Year 1 (\$32,340). In Year 2, 4 students will again each work 4 weeks in the field in the summer  (35 hours per week, per student) and 24 weeks in the lab for the soil pathogen study at roughly 20 hour per week on average (some weeks of prep in the summer will be higher, with lower hours each week during term when the focus will be on monitoring), plus an additional 8 weeks (at 20 hours per week on average) to begin prep work for DNA extraction, for a total of 3120 undergraduate hours (4 people x 4 weeks in the field x 35 hours per week plus 4 people x 24 weeks in the lab x 20 hours on average per week plus 4 people x 8 weeks in the lab x 20 hours on average per week) in Year 2 (\$40,560). In Year 3 (final year), 3 students will each work 4 weeks in the field in the summer  (35 hours per week, per student) and 32 weeks in the lab for the soil pathogen study at roughly 20 hour per week on average which will be mainly for DNA extraction, purification and quantification (which substantially reduces the costs of sequencing, while giving students additional training) for a total of 2340 undergraduate hours (3 people x 4 weeks in the field x 35 hours per week plus 3 people x 32 weeks in the lab x 20 hours on average per week) in Year 3 (\$30,420). 

{\bf Equipment or facility:}\\
The lab will purchase 680 tree dendrometers and associated electronic logging systems at \$42.735 per unit for a total system cost of \$29,060 in Year 1 to measure tree growth at Mount Rainier. These will be installed in Year 1. In Year 3, the lab requests costs for sequencing (EZ-16S and EZ-ITS) for 500 samples from the soil pathogen studies, estimated at \$103.16 per sample (this rate assumes we send extracted, purified and quantified DNA) and returns sequencing results with basic pipeline output (basic taxonomy and diversity statistics), for a total cost of \$51,580 in Year 3.  

{\bf Materials, Supplies \& User Fees}\\
The lab has most other major equipment needed (e.g., scales, tree corers, tree core imaging, centrifuge) thus support is requested to cover costs associated with the soil pathogen study for supplies and minor tools for DNA extraction, purification and quantification (we expect to have chamber fees for the soil pathogen study waived given the chambers were purchased through the lab's CFI grant recently), and related Materials and Supplies as follows:

\underline{Basic materials for soil pathogen assays} including pots, soils, gloves and tubes to freeze soils is requested at \$1,700 in Year 1 and slightly lower (assuming we will re-use some materials, as we usually can) at \$1,400 in Year 2. 

\underline{Materials for DNA extraction, purification and quantification} including pipettes, extraction kits (current estimated costs are for Qiagen DNeasy PowerSoil kits) and various tubes and containers estimated at \$10,000 in Year 2 when we will begin DNA work and \$7,500 in Year 3 when we will complete DNA work and send samples for sequencing. 

{\bf Travel}

\emph{Travel to professional meetings}\\
I plan to support my own conference attendance and conference attendance for MSc and PhD students (and possibly undergraduates, depending on their interests) through professional funds available at UBC and my CRC. % with additional support requested in years 3 and 5 (one conference each year) that I hope will be used by undergraduates

\emph{Travel to visit collaborators}\\
Travel in Year 1 is focused on supporting travel to ETH for 1 member of my lab (likely Xiaomao Wang), estimated at \$3,000 including base airfare and housing. In Year 2, travel costs will support a larger team of 1 PhD student, 1 MSc student and PI Wolkovich to visit ETH for a slightly longer period, with an estimated cost of \$14,000. In Year 3,  travel funds will support 1 PhD student and PI Wolkovich to again visit ETH, with an estimated cost of \$10,500. \\

{\bf Dissemination}\\ 
CRC funds in the lab are expected to cover publication costs and other related dissemination. \\


\end{document}