\documentclass[12pt,oneside]{article}
\renewcommand{\baselinestretch}{1.8}
% \renewcommand*{\thefootnote}{\fnsymbol{footnote}}
\usepackage{sectsty,setspace} 
\usepackage[top=2.1cm, bottom=2.1cm, left=2.1cm, right=2.1cm]{geometry} 
\usepackage{graphicx}
\usepackage{epstopdf}
\usepackage{amsmath,latexsym,amssymb,wasysym}
\usepackage{cite}
\usepackage{citesupernumber}
%\usepackage{citecollapse}
\usepackage{hyperref}
\usepackage{float}
\floatstyle{plaintop}
\restylefloat{table}
\usepackage[table]{xcolor} % http://ctan.org/pkg/xcolor
\usepackage{mathptmx} % closest to Times New Roman (https://tex.stackexchange.com/questions/153168/how-to-set-document-font-to-times-new-roman-by-command)

\setlength\parindent{0pt} % no indents throughout

\setlength{\abovecaptionskip}{1pt}
\setlength{\belowcaptionskip}{-10pt}
\usepackage[font={small}]{caption}

\parskip=5pt
\pagenumbering{arabic}
\pagestyle{plain}
% squeeze spacey
\linespread{0.99}
\addtolength{\itemsep}{-0.05in}
 
\newenvironment{smitemize}{
\begin{itemize}
  \setlength{\itemsep}{1pt}
  \setlength{\parskip}{0pt}
  \setlength{\parsep}{0pt}}
{\end{itemize}
}

\usepackage{fancyhdr}
\pagestyle{fancy}
\fancyhead[LO]{NSERC Alliance Intl 2024}
\fancyhead[RO]{E. M. Wolkovich}

\newcommand{\Section}[1]{\vspace{-8pt}\section{\hskip -1em.~~#1}\vspace{-3pt}} 

\begin{document}

% https://www.nserc-crsng.gc.ca/OnlineServices-ServicesEnLigne/instructions/101/allianceinternational_eng.asp
% You need to include the TEXT of what they want in the 3 pages; most examples have ONE page for the first section; and most have 0.5 for EDI

\bibliographystyle{/Users/Lizzie/Documents/EndnoteRelated/Bibtex/styles/ajhg}
\thispagestyle{empty}
\begin{center}
{\bf How temporal and spatial mosaics of reproduction determine forest communities} 
\vspace{-1ex}
\end{center}
% Alt title: How pulsed reproduction shapes temperate forest communities 
% SNSF title: Costs and Benefits of Masting in Temperate Forest Communities (MastFor)

% Summary of proposal
% Write a summary of the proposed research, intended to explain the proposal in language that the public can understand.
Temperate forests are a critical global resource and one especially important to Canada---supporting the forestry, maple syrup, and related industries while providing numerous additional services to people. Their persistence and regeneration shapes the global carbon cycle of today and our future. Yet how forest tree communities assemble and are maintained is still a major question in ecology---and one that is more pressing to answer given human-caused climate change. Addressing this question requires understanding how tree seeds and seedlings successfully navigate a world of seed predators (including small mammal and birds) and soil pathogens to become adult trees. This project examines how pulsed tree seed production (called `masting') and spatial mosaics of seed and seedling death may promote forest regeneration. Through a collaborative Canadian-Swiss alliance of two major forest ecology labs, this project will gather data across a range of forest sites---across two countries and three years---to test the relative roles of seed predators and pathogens in determining seed and seedling survival over both space and time. Complementary skills from each team will allow the project to build population models of forests that can test competing hypotheses for what drives assembly and provide future predictions. By working across a climatic gradient just south of BC and incorporating tree growth, this project will help improve predictions of the assembly and resilience of Canadian forests as the climate warms, with implications for the global carbon cycle. 
% Through a collaborative Canadian-Swiss alliance of two major forest ecology labs this project will gather data across temperate forests to test multiple processes across space and time at once. Working together, the international team will gather data across a range of forests sites---across two countries and three years---to test the relative roles of seed predators and pathogens in determining seed and seedling survival over both space and time. Complementary skills from each team will allow the project to build population models of forests that can test competing hypotheses for what drives assembly and provide future predictions. The project is designed to provide a path towards a united model of forest assembly that includes complex processes across space and time that likely shape successful tree establishment.  By working across a climatic gradient just south of BC and incorporating tree growth, this project will help improve predictions of the assembly and resilience of Canadian forests as the climate warms, with implications for the global carbon cycle. 

% While simplistically communities dynamics should be predictable from how trees grow and reproduce, the pathway from growth to seed production to seeds successfully navigating a world of seed and seedling predators--for example, squirrels and bird---pathogens, including myriad viruses and bacteria is an open question in ecology. This project examines two major models of forest growth and reproduction---one focused on pulsed tree seed production and another focused on spatial mosaics of seed and seedling death.
\newpage
\setcounter{page}{1}

% From proposal template ... 3 pages MAX

{\sc Relevance and expected outcomes} 
\vspace{-1ex}
\begin{smitemize}
\item Outline the objectives of the proposed international collaboration and explain the potential outcomes and impacts. Explain how the international collaboration will address important research challenges in the natural sciences and engineering (NSE) disciplines and further develop areas of Canadian strength and leadership. 
\item Describe the global importance of the topic and how the expected outcomes could benefit Canada.
\end{smitemize}
Understanding the major factors that shape forest communities is a fundamental question across ecology and forest science, with decades of research yielding competing hypotheses. Most hypotheses focus on the challenge of understanding how seeds and seedlings survive a world of predators and pathogens to become saplings.\cite{janzen1971seed,connell1983prevalence,comita2014testing,davies2024seed} This project unites and tests the two major models for forest regeneration from seed in temperate forests: (1) pulsed seed production in only certain years---called `masting'---as a way to temporally structure seed predators such that they are at low abundance in high seed years;\cite{koenig2021brief,pearse2016mechanisms} and (2) spatial negative density-dependent survival of seeds and seedlings\cite{connell1983prevalence,comita2014testing}---often called `Janzen-Connell'---caused by hypothesized high predator or pathogen density near a parent tree that declines with distance from the parent tree, creating spatial mosaics in survival for seeds.\cite{janzen1970herbivores,connell1983prevalence} While both hypotheses focus on the challenge of seed and seedling survival as the major mystery of forest regeneration, they operate on contrasting axes---with masting focused on temporal patterns of recruitment,\cite{pearse2017inter} and Janzen-Connell focused on spatial patterns. Importantly, they also make contrasting predictions. Masting uses an economy of scale---high seed production can swamp predator populations, assuming most seed sources for the predators mast at the same time---to predict positive density dependence (more seeds lead to more survival), while Janzen-Connell predicts negative density dependence. They also make varying predictions for how environmental factors affect seed production and in turn adult tree growth. Masting suggests plants use environmental cues to time high-reproduction years and experience associated growth declines due to high investment in reproduction.\cite{koenig1998scale,hacket2016tree,pearse2016mechanisms,bogdziewicz2021climate} In contrast, Janzen-Connell requires no environmental cues and predicts more regular reproduction with less noticeable effects on growth. 

We argue that both mechanisms for forest recruitment---masting and Janzen-Connell---may explain recruitment in temperate forests. While generally considered separately,\cite{bogdziewicz2024evolutionary} testing both mechanisms at once may show that they operate together in complementary ways, which could explain existing patterns in tree regeneration. Our project leverages collaboration between two forest ecology labs to build on existing data and collect the additional data critical to test these models together. 
% The international collaborator will importantly provide long-term data temporally and spatially explicit data on seed production and seedling survival that is rare and required to address these questions, while the Canadian collaborator will bring expertise in Bayesian modeling to address these two hypotheses together, and collect the tree growth data that is necessary to predict long-term forest dynamics, as well as oversee field assays of predators and soil pathogens. 
Globally the need for a robust mechanistic model of forest tree dynamics is critical to the global carbon cycle, with Canada's forests playing a major role.\cite{ipcc2021,friedlingstein2022global} In Canada, this is also an especially pressing challenge as industries related to forests (e.g., tourism, wood products, maple syrup) are critical to the economy and at risk of major shifts with increased warming from human-caused climate change. With improved models of forest dynamics and how they respond to environmental cues, including regeneration from seed---a major focus of this work---Canada will be better poised to mitigate major forest change when possible (for example, through targeted seed or planting programs in actively managed forests), and mitigate impacts. 

% ... and one made more pressing by climate change. 

{\sc Collaboration:} % 0.75 pages
\vspace{-1ex}
\begin{smitemize}
\item List and describe your international collaborator(s).
\item Explain the rationale for your selected international collaborator(s) and the added value of the collaboration.
\item Describe the role of your team in the collaboration.
\end{smitemize}
% Copied from another proposal: This project allows two emerging research groups to integrate their unmatched expertise towards paradigm-changing answers 
% This collaboration will help to strengthen international scientific networks. 
The international collaborator, Janneke Hille Ris Lambers, is a world expert in forest ecology,\cite{clark1999seed,hille2005implications,ettinger2013climate,hillerislambers2013will,ford2017competition,buckley2019temperate,ford2020soil} with special expertise in global change ecology of temperate forests---including Mount Rainier (called Tahoma by local indigenous groups)---where she has worked for over 15 years.\cite{ettinger2011climate,anderegg2016drought,ford2017competition,ford2020soil} Her research focus on forest assembly, including the role of Janzen-Connell dynamics,\cite{hille2002density} makes her uniquely skilled to co-lead this research. % and the Temporal Ecology lab, run by Wolkovich, will gain from her knowledge, methods, skills and data. 
% She is currently Professor and Group Leader at ETH-Zurich, where she runs the Plant Ecology Group of roughly 8-12 trainees (master and doctoral students and postdocs) and 10 research staff, and has now studied Swiss temperate forests for over four years. As detailed further in the accompanying biographical sketch, Hille Ris Lambers is one of the most respected researchers in both forest ecology and global change, and has reshaped how ecologists think about temperate forest assembly. 
% The by project is centered around collaboration from two forest ecology labs. As outlined in the funded SNSF, the project is impossible without the combined skills, data and collaboration of the two labs. 

This proposed work is large in scope---designed to potentially reshape how we understand forest dynamics in temperate forests---and is only possible through close collaboration between Dr. Hille Ris Lambers, ETH, and my own research group at UBC. Each of the proposal's three work packages (WP) require integrated collaboration between the two labs. Dr. Hille Ris Lambers' group provides rare spatially and temporally rich data on seed production and seedling recruitment, while my lab will collect tree growth data through tree cores in the same locations to round out the data required to test the costs and benefits of each hypothesis (outlined in WP1). New studies to assess the role of mammalian seed predators and soil pathogens in structuring seed and seedling dynamics will be carried out by each lab in WP2, with my lab leading the soil microbial sequencing aspect of WP2 for US sites. 

Both PIs are skilled in modeling complex ecological dynamics, which will be critical for all WPs. Wolkovich is especially adept at Bayesian approaches that can be used for predictions,\cite{ospreebbms,dan2021je,decsens,morales2024phylogenetic} while Hille Ris Lambers brings expertise in demographic models,\cite{hille2002density,clark2003estimating,ford2020soil} two skillsets that come together in WP3 where predictive models will estimate demographic costs and benefits of masting. These models can help uncover gaps in each hypothesis---for example by showing cases where model predictions do not match existing patterns in the data---and can be adapted to make useful forecasts of how forests may change with continued climate change.\cite{IPCC:2014sm} % This could be especially beneficial in Canada where climate change is expected to accelerate in the coming decades and forecasts of forest change could help managers mitigate and adapt to climate change. 

This collaboration will strengthen an international scientific network critical for adapting to continued global change. In addition to the exchange for trainees directly involved in the research (outlined in the Training plan below), this project will bring important skills in how to measure and model forest demography for forecasting to UBC and Canada. The proposed US sites are located just south of BC (Mount Rainier), and thus may serve as a sentinel of what warming will bring to BC forests. Further, my research group will learn first-hand the forest-demography methods used by our international collaborator, and will then have the skills to gather similar data for BC. % Already, the Temporal Ecology Lab has leveraged some of the Plant Ecology Group's methods to collect one aspect of long-term forest demography data (seedling counts) in forests along an elevational gradient Smithers, BC, and this collaboration would build the network, that could lead eventually to research across US and Canadian sites.  
%% CITE the hell out of Janneke and my work HERE!! %% 

{\sc Training plan:}  % 0.75 pages
\vspace{-1ex}
\begin{smitemize}
\item  Describe how the project and the international collaboration offer opportunities for enriched training experiences that will allow research trainees (undergraduates, graduates and postdoctoral fellows) to develop relevant technical skills as well as professional skills, such as leadership, communication, collaboration and entrepreneurship. Include the nature of the planned interactions with the international collaborators(s) and other relevant activities.
\end{smitemize}
% Copied from another proposal: This collaboration will empower HQP to develop the skills typically emphasized in research and coursework within a more high-stakes environment, where deliverables have immediate real-world consequences. ... The Canadian portion of this proposal will specifically support the effort of 2 PhD students... 
% Should mention github etc.; monthly meetings; international collaborative meetings in Vancouver, Washington State and Switzerland ... 
This collaboration will build an international network of trainees skilled in field and lab forest demography and ecology methods alongside robust computational and analytical approaches. As outlined in the SNSF grant, Canadian HQP are integral to the proposed research. A current UBC PhD student (Xiaomao Wang) will play a critical role in collecting and analyzing tree growth data. This student will be joined by a MSc student at UBC who will focus on seed predator trials and help with soil pathogen assays (WP2) with support from the proposed Swiss postdoc and additional support through my own research group. All students will be supported by and help mentor 8-22 undergraduate students on the project, who will assist in field and lab work while taking on specific components of a task to gain experience in project design. This team will work alongside a Swiss team spanning similar career stages with a project designed around team collaboration and training.

Both Wolkovich and Hille Ris Lambers are committed to trainees on this project becoming well-rounded professionals through an enriched international program of meetings and skill building. The labs plan in-person meetings each year  alongside bi-monthly videocalls to maintain momentum, troubleshoot issues and provide additional support as needed. In-person meetings will have focused training components, including forest ecology and demography field methods and data science skills for reproducible research in the first year. Meetings in following years will focus on analytical methods and will provide  opportunities for trainees to learn Bayesian methods and the language Stan from Wolkovich, and demographic modeling from Hille Ris Lambers. All meetings will include structured and unstructured opportunities to build collaboration, leadership and communication skills as different trainees will lead different aspects of the project and present their progress regularly. In Year 3, these meetings will include time to practice presentations for scientific meetings, alongside career discussions of next steps and opportunities.  % The Wolkovich lab holds career-focused lab meetings at least once each year and the opportunity to expand these meetings to include a more international perspective will benefit the entire Temporal Ecology Lab. % including an introduction to git and GitHub as well as data management skills.

{\sc Equity, diversity, and inclusion (EDI):}  % 0.75 pages
\vspace{-1ex}
\begin{smitemize}
\item  Describe how EDI will be fostered within the research environment, including 1) how EDI will be considered in the training plan and 2) how EDI has been considered in the academic team composition.
\end{smitemize}
Both myself and Dr. Hille Ris Lambers recognize how much academia's failure to attract, retain and promote the full diversity of people in our broader communities limits our research, outreach and makes our related work less useful. Both labs are thus committed to actionable steps to change this reality, and have a history of efforts lower barriers to entry and success, which they will build upon in this project. To increase the diversity of HQP, all vacant positions will be open with clear job descriptions that encourage candidates with diverse backgrounds to apply and are reviewed by colleagues specializing in this before being shared widely on job boards and listservs.  The PI and at least two lab members will review all candidates with a rubric designed to minimize implicit bias. My group maintains a suite of additional guidelines for equitable hiring and science as discussed in the Personal data form. % My lab maintains a tumblr so interested candidates can gain a sense of what joining the lab will be like, and we keep an evolving website and code of conduct publicly available that we link to in all ads. We review and update these documents at least once a year as part of regular lab meetings on research and issues slowing broader participation. Our github also includes explicit expectations for open data and sharing skills and lab tasks, to make sure work and knowledge are shared equitably.  

Work in the field, lab and computationally will also be designed to foster a welcoming and inclusive environment, and to distribute training opportunities equitably. At the field and lab work stages, the PIs will carefully design tasks so they are approachable for trainees. This includes, for example, using drills or starters for coring trees (WP1) to reduce the physical force needed.  Wolkovich will work together with all trainees on an Individual Safety Plan (ISP) so they are aware and prepared for any hazards at Mount Rainier.  Computational work and training will be carefully shared and managed, with trainees from undergraduate to PhD levels given similar structured training in data management, reproducible science and---for MSc, PhD and interested undergraduates---training in Bayesian modeling. I make sure all trainees have equal access to training opportunities that vary in style to meet different learning needs and styles and help them build a network of support (e.g., customized trainings in the lab, workshops offered by different groups and societies, on-line classes from experts who specialize in Bayesian methods). % For example, Michael Betancourt, a Latino-American computational statistician who works on increasing diversity in Bayesian Statistics will visit the Temporal Ecology lab to offer training and one-on-one meetings for this project specifically. 

Team meetings will be carefully designed to make sure all lab members can attend and feel fully welcome. Both labs have a proven track record of designing meeting times, locations and structures to support lab members' family, religious, health-related and other life commitments and challenges while also giving them a robust opportunity for collaboration, team science and training. This means the PIs will work out meeting schedules well in advance, carefully choose locations and activities that are inclusive for all and make sure trainees feel they can share concerns. Discussing how to make team meetings equitable and inclusive  will be part of the early video-call meetings, including structured readings and discussions as needed. % While both labs are an active work-in-progress towards a more inclusive and equitable system in academia, both PIs efforts towards this are clearly recognized: Wolkovich currently serves on the Steering Committee for Harnessing Diversity in Data Science and Hille Ris Lambers ... XX %  and the two labs look forward to working together to make this work for an international team and expect each group will learn new techniques to promote meetings. 


\clearpage
\setcounter{page}{1}
{\bf References:}
{\def\section*#1{}
\bibliography{nseralliancerefs.bib}}

\end{document}

References % 2 pages good, one example had 2.25 pages, one example had one page
•	Use this section to provide a list of the most relevant literature references. Do not refer readers to websites for additional information on your proposal. Do not introduce hyperlinks in your list of references.
•	These pages are not included in the page count.
%  by two proposed MSc students (or one PhD student, depending on candidates who apply to open calls for this position, see also EDI section below)

% Need budget and justification (separate from 3 page max)
% Should add cover letter -- see examples/Development of forecast, analytic, and visualization tools to improve outbreak response and support public health decision-making for one that looked pretty good. 


International collaborator(s) and collaborator biographical sketches
Indicate the proposed international collaborator(s). The collaborator will not have access to the grant funds and must be qualified to undertake research independently. For Collaboration grants, your international collaborator(s) must have secured funding for their part of the project prior to you submitting your application to NSERC. On behalf of your collaborator(s), attach a biographical sketch. In a maximum of two pages for each collaborator, provide their name/affiliation, education/training, employment/affiliations, research funding and up to five significant contributions related to the project.

