\documentclass[11pt]{article}
\usepackage[top=1.00in, bottom=1.0in, left=1in, right=1in]{geometry}
\renewcommand{\baselinestretch}{1.1}
\usepackage{graphicx}
\usepackage{natbib}
\usepackage{amsmath}
\usepackage{parskip}

\def\labelitemi{--}
\parindent=0pt

\begin{document}
\bibliographystyle{/Users/Lizzie/Documents/EndnoteRelated/Bibtex/styles/besjournals}
\renewcommand{\refname}{\CHead{}}

\setlength{\parindent}{0cm}
\setlength{\parskip}{5pt}

% https://www.nserc-crsng.gc.ca/OnlineServices-ServicesEnLigne/instructions/101/allianceinternational_eng.asp

% You need to include the TEXT of what they want in the 3 pages; most examples have ONE page for the first section; and most have 0.5 for EDI

Title: How pulsed reproduction shapes temperate forest communities 
% SNSF title: Costs and Benefits of Masting in Temperate Forest Communities (MastFor)

% Summary of proposal
% Write a summary of the proposed research, intended to explain the proposal in language that the public can understand.

% Temperate forests are a critical global resource and one especially important to Canada. They are a major component of the industries around forestry, including wood products and maple syrup, provide numerous additional services to people and shape the global carbon cycle. How forest tree communities form and persist, however, is still a major question in ecology---and one that is more pressing to answer given human-caused climate change, which is already changing our forests. While simplistically communities dynamics should be predictable from how trees grow and reproduce, the pathway from growth to seed production to seeds successfully navigating a world of seed and seedling predators--for example, squirrels and bird---pathogens, including myriad viruses and bacteria is an open question in ecology. This project examines two major models of forest growth and reproduction---one focused on pulsed tree seed production and another focused on spatial mosaics of seed and seedling death.


% From proposal template ... 3 pages MAX

{\bf Relevance and expected outcomes} % 0.75 pages
% •	Outline the objectives of the proposed international collaboration and explain the potential outcomes and impacts.
% •	Explain how the international collaboration will address important research challenges in the natural sciences and engineering (NSE) disciplines and further develop areas of Canadian strength and leadership. Describe the global importance of the topic and how the expected outcomes could benefit Canada.

Understanding the major factors that shape forest communities is a fundamental question across ecology and forest science, with decades of research producing competing hypotheses that have yet to produce one robust mechanistic model. Most hypotheses focus on the challenge of understanding how seeds and seedlings survive a world of pests, predators and pathogens to become saplings, at which point light gaps and other more stochastic processes seem to determine which individuals become canopy trees. 

This project unites and tests the two major models for forest regeneration from seed in temperate forests: (1) pulsed seed production in only certain years---called `masting' or `mast seeding' usually---as a way to temporally structure seed predators such that they are at low abundance in high seed years; and (2) spatial negative density-dependent survival of seeds and seedlings. This later mechanism, often called `Janzen-Connell' effects, is caused by hypothesized high predator or pathogen presence near a parent tree that declines with distance from the parent tree, creating spatial mosaics in survival for seeds. While both hypotheses focus on the challenge of seed survival as the major mystery of forest regeneration, they operate on contrasting axes---with masting focused on temporal patterns of recruitment, while Janzen-Connell focuses on spatial patterns---and make somewhat contrasting predictions. Masting uses an economy of scale---high seed production can swamp predator populations assuming most seed sources for the predators mast (produce an abundant seed crop) at the same time---to predict positive density dependence (more seeds lead to more survival), while Janzen-Connell predicts negative density dependence. In Janzen-Connell, seed production is highest near the parent tree where survival is also lowest and maximum recruitment is predicted where seed production is lower. They also make varying predictions for how environmental factors effect seed production and in turn adult tree growth; masting suggests plants use cues to time high-production years and would experience associated growth declines due to high investment in reproduction, while Janzen-Connell predicts more regular reproduction with less noticeable effects on growth and due to the environment. 

We argue both mechanisms for forest recruitment---masting and spatial negative density-dependent survival---may explain recruitment in temperate forests, but has rarely been tested. Our project leverages collaboration between two forest ecology labs to build on existing data and collect the additional data critical to testing these models together. The international collaborator will importantly provide long-term data temporally and spatially explicit data on seed production and seedling survival that is rare and required to address these questions, while the Canadian collaborator will bring expertise in Bayesian modeling to address these two hypotheses together, and collect the tree growth data that is necessary to predict long-term forest dynamics. 

Globally the need for a robust mechanistic model of forest tree dynamics is critical to the global carbon cycle, with Canada's forests playing a major role. In Canada, this is also an especially pressing challenge as industries related to forests (e.g., tourism, wood products, maple syrup) are critical to the economy and at risk of major shifts with increased warming from human-caused climate change. With improved models of forest dynamics and how they respond to environmental cues, including regeneration from seed which is a major focus of this work, Canada will be better poised to mitigate major forest change when possible (for example, through targeted seed or planting programs in actively managed forest), and mitigate impacts of forest change (for example, through understanding impacts on small mammal and bird communities from this work). 

% ... and one made more pressing by climate change. 

{\bf Collaboration:} % 0.75 pages
% •	List and describe your international collaborator(s).
% •	Explain the rationale for your selected international collaborator(s) and the added value of the collaboration.
% •	Describe the role of your team in the collaboration.
% Copied from another proposal: This project allows two emerging research groups to integrate their unmatched expertise towards paradigm-changing answers 
% This collaboration will help to strengthen international scientific networks. 
International Collaborator: Janneke Hille Ris Lambers is da bomb ... She is a forest ecology expert ... global change expert.

% Go through WP1, WP2, WP3 here and outline the collaboration
This project is centered around collaboration from two forest ecology labs. As outlined in the funded SNSF, the project is impossible without the combined skills, data and collaboration of the two labs. The Hille Ris Lambers lab provides rare spatially and temporally rich data on seed production and seedling recruitment to test the two major hypotheses for forest regeneration, while the Wolkovich lab will collect tree growth data through tree cores in the same locations to round out the data required to test the costs and benefits of each hypothesis. Both are skilled in modeling complex ecological dynamics, with Wolkovich especially adept at building Bayesian models that can be used for predictions. These predictions can help uncover gaps in each hypothesis---for example by showing cases where model predictions do not match existing pattens in the data---and can be adapted to make useful forecasts of how forest may change with continued climate change. This could be especially beneficial in Canada where climate change is expected to accelerate in the coming decades and forecasts of forest change could help managers mitigate and adapt to climate change. 

The sites are located close to BC (Mount Rainier) and the Wolkovich lab already has begun to collect long-term forest demography data in BC forests in Smithers. These data could totally help BC .... 

And we have been collaborating over two years ...
 
CITE the hell out of Janneke and my work HERE.

Training plan % 0.75 pages
% Copied from another proposal: This collaboration will empower HQP to develop the skills typically emphasized in research and coursework within a more high-stakes environment, where deliverables have immediate real-world consequences. ... The Canadian portion of this proposal will specifically support the effort of 2 PhD students... 
% Should mention github etc.; monthly meetings; international collaborative meetings in Vancouver, Washington State and Switzerland ... 
•	Describe how the project and the international collaboration offer opportunities for enriched training experiences that will allow research trainees (undergraduates, graduates and postdoctoral fellows) to develop relevant technical skills as well as professional skills, such as leadership, communication, collaboration and entrepreneurship. Include the nature of the planned interactions with the international collaborators(s) and other relevant activities.

INSERT YOUR TEXT HERE, RESPONDING TO EACH OF THE ABOVE POINTS

Equity, diversity, and inclusion (EDI) % 0.75 pages
•	Describe how EDI will be fostered within the research environment, including 1) how EDI will be considered in the training plan and 2) how EDI has been considered in the academic team composition.

INSERT YOUR TEXT HERE, RESPONDING TO EACH OF THE ABOVE POINTS

References % 2 pages good, one example had 2.25 pages, one example had one page
•	Use this section to provide a list of the most relevant literature references. Do not refer readers to websites for additional information on your proposal. Do not introduce hyperlinks in your list of references.
•	These pages are not included in the page count.


% Need budget and justification (separate from 3 page max)
% Should add cover letter -- see examples/Development of forecast, analytic, and visualization tools to improve outbreak response and support public health decision-making for one that looked pretty good. 

\end{document}

International collaborator(s) and collaborator biographical sketches
Indicate the proposed international collaborator(s). The collaborator will not have access to the grant funds and must be qualified to undertake research independently. For Collaboration grants, your international collaborator(s) must have secured funding for their part of the project prior to you submitting your application to NSERC. On behalf of your collaborator(s), attach a biographical sketch. In a maximum of two pages for each collaborator, provide their name/affiliation, education/training, employment/affiliations, research funding and up to five significant contributions related to the project.

